\documentclass[11pt]{amsart}
\usepackage{geometry}
\geometry{letterpaper}
\usepackage{graphicx}
\usepackage{amssymb}
\usepackage{mathrsfs}

\title{A Tutorial Introduction to Scaling, Bifurcation Analysis and Chaos:\\
Laser Rate Equations and Beyond}
\author{Prof. Robert Flassig}
\date{\today}

\begin{document}
\maketitle
\tableofcontents
\section{Introduction}

Many nonlinear dynamical systems in physics contain parameters with physical
units.  Before analyzing equilibria, bifurcations, or stability, it is best
practice to \emph{nondimensionalize} the equations: convert them to a form
involving only dimensionless variables and dimensionless parameters.  
This simplifies the mathematics, exposes universal behavior, and reduces the
number of relevant parameters.

In this tutorial we will:

\begin{enumerate}
    \item review best practices for nondimensionalization with a simple example,
    \item apply these ideas to the semiclassical laser rate equations,
    \item perform a fixed point and stability analysis,
    \item sketch the bifurcation structure of the laser threshold,
    \item briefly discuss how one might extend the system toward chaotic behavior.
\end{enumerate}

Note: Equations and plots can be found here: 

\vspace{1em}
\section{A Brief Reminder: How to Nondimensionalize a System}

Consider a system written in dimensional variables,
\[
\dot x = f(x,t; \text{parameters}).
\]
To nondimensionalize, we choose characteristic scales:
\[
x = X_0 \,\tilde x, \qquad t = T_0 \,\tilde t,
\]
where $X_0, T_0$ have the same units as $x$ and $t$ and are constructed from the
physical parameters.  After substituting and dividing out physical units, we
obtain an equation of the form
\[
\tilde x' = F(\tilde x; \text{dimensionless parameters}).
\]

\subsection{Simple Example}

Take the equation
\[
\dot x = ax - bx^2, \qquad a,b>0.
\]
The natural scale for $x$ is the steady state $x^\ast = a/b$, so write
\[
x = \frac{a}{b}\,\tilde x.
\]
The natural time scale is $1/a$, so write $t = \tilde t / a$.  
Substituting yields
\[
\tilde x' = \tilde x - \tilde x^2.
\]
All parameters have disappeared: the model is now universal.
This is considered \emph{best practice}: identify natural physical scales coming
from steady states, decay rates, or characteristic times, and scale variables
accordingly.

\vspace{1em}
\section{Laser Rate Equations}

We now consider the semiclassical laser equations
\begin{align}
\dot n &= G\, n\, N - k n , \label{eq:n}\\[2mm]
\dot N &= - G\, n\, N - f N + p , \label{eq:N}
\end{align}
where:
\[
n(t)=\text{photon number},\qquad
N(t)=\text{excited atoms},
\]
\[
G > 0 \text{ gain}, \quad
k>0 \text{ cavity loss}, \quad
f >0  \text{ spontaneous decay}, \quad
p\in\mathbb{R} \text{ pump}.
\]

\section{Nondimensionalization}

A natural time scale is the photon decay time $1/k$, so we define
\[
\tau = k t.
\]
For $n$ and $N$, we choose scales suggested by the nonlinear coupling $G n N$:
\[
x = \frac{G}{k}n, \qquad y = \frac{G}{f}N.
\]
Finally define dimensionless parameters
\[
\alpha = \frac{f}{k}, \qquad
\beta  = \frac{G}{kf}p.
\]

Using derivatives with respect to $\tau$ (denoted by $'$), the system becomes
\begin{align}
x' &= x (y - 1), \label{eq:x}\\
y' &= - \alpha y - \alpha x y + \alpha \beta. \label{eq:y}
\end{align}
This is a two-dimensional nonlinear system with only two parameters
$\alpha>0$ and $\beta\in\mathbb{R}$.

\vspace{1em}
\section{Fixed Points and Their Classification}

To find equilibria, solve $x'=0$ and $y'=0$.

\subsection{(a) Trivial equilibrium}

If $x=0$, then $y' = -\alpha y + \alpha \beta = 0$ gives
\[
E_0 = (0, \beta).
\]
This corresponds to zero photons in the cavity.

\subsection{(b) Lasing equilibrium}

If $y=1$ (from $x'=0$), substitute into $y'=0$:
\[
0 = -\alpha(1+x) + \alpha \beta
    \quad\Rightarrow\quad
x = \beta - 1.
\]
Thus, a second equilibrium exists:
\[
E_L = (\beta-1,\,1), \qquad \text{only if } \beta > 1.
\]
This is the operating point of a laser above threshold.

\subsection{Jacobian and Stability}

The Jacobian is
\[
J(x,y)=
\begin{pmatrix}
y-1 & x \\
-\alpha y & -\alpha(1+x)
\end{pmatrix}.
\]

\paragraph{Stability of $E_0=(0,\beta)$.}
\[
J(E_0)=
\begin{pmatrix}
\beta-1 & 0 \\
-\alpha\beta & -\alpha
\end{pmatrix}.
\]
Eigenvalues:
\[
\lambda_1=\beta-1,\qquad \lambda_2=-\alpha.
\]
Thus
\[
\begin{cases}
\beta<1 &: \text{$E_0$ is a stable node},\\
\beta=1 &: \text{$E_0$ is nonhyperbolic (bifurcation point)},\\
\beta>1 &: \text{$E_0$ becomes a saddle}.
\end{cases}
\]

\paragraph{Stability of $E_L = (\beta-1,1)$.}

\[
J(E_L)=
\begin{pmatrix}
0 & \beta-1\\
-\alpha & -\alpha\beta
\end{pmatrix}.
\]

Eigenvalues satisfy
\[
\lambda^2 + \alpha\beta\,\lambda + \alpha(\beta-1)=0.
\]
The real part is always negative for $\beta>1$, so $E_L$ is
\emph{always stable} (node or spiral).

The discriminant
\[
\Delta=(\alpha\beta)^2-4\alpha(\beta-1)
\]
determines node vs.\ spiral.

\vspace{1em}
\section{Phase Portraits and Qualitative Dynamics}

The system exhibits qualitatively different behavior depending on $\beta$:

\begin{enumerate}
\item \textbf{Subthreshold ($\beta<1$):} only $E_0$ exists and is stable.
Photon number decays to zero.

\item \textbf{Threshold ($\beta=1$):} a \emph{transcritical bifurcation}.  
The laser turns on at this point.

\item \textbf{Above threshold ($\beta>1$):} $E_0$ becomes a saddle and $E_L$ becomes stable.  
Approach to $E_L$ may be monotone (node) or oscillatory (spiral).
\end{enumerate}

\vspace{1em}
\section{Stability Diagram and Bifurcation}

The curve separating nodes from spirals is given by $\Delta=0$:
\[
\alpha = \frac{4(\beta-1)}{\beta^2}.
\]
Below this curve: stable spiral (damped laser oscillations).  
Above this curve: stable node (non-oscillatory relaxation).

The line $\beta=1$ is the \emph{laser threshold bifurcation}, a classical
\textbf{transcritical bifurcation}.

\vspace{1em}
\section{Can the Laser System Become Chaotic?}

The nondimensional laser rate equations
\[
x' = x (y - 1), \qquad 
y' = - \alpha y - \alpha x y + \alpha \beta,
\]
form a two–dimensional autonomous dynamical system.  
By the Poincaré–Bendixson theorem, such systems cannot exhibit deterministic
chaos: their long–term behavior consists only of fixed points or limit cycles.
Thus, \emph{the simplest semiclassical laser model cannot become chaotic}.

Nevertheless, real lasers \emph{do} display chaotic intensity fluctuations, and
several extensions of the basic model lead naturally to chaos.  The most
famous example is the \emph{Haken–Lorenz laser model}, which has exactly the
same mathematical structure as the classical Lorenz equations. 

\subsection{A Concrete Example: The Haken--Lorenz Equations}

To move from the 2--variable model to a chaotic one, we include a third
dynamical variable representing the \emph{polarization} of the atomic medium.
Typical laser physics leads to the three coupled equations
\begin{align}
\dot E &= \kappa (P - E), \label{HL1}\\[2mm]
\dot P &= \gamma (E N - P), \label{HL2}\\[2mm]
\dot N &= \rho (\beta - N - E P), \label{HL3}
\end{align}
where:

\[
E = \text{electric field amplitude}, \quad 
P = \text{medium polarization},\quad 
N = \text{population inversion}.
\]

The parameters $\kappa, \gamma, \rho$ measure the decay rates of the field,
polarization, and inversion, respectively.  After suitable nondimensionalization
(dropping tildes), the system can be cast into the form
\begin{align}
\dot X &= \sigma (Y - X), \\
\dot Y &= r X - Y - X Z, \\
\dot Z &= X Y - b Z,
\end{align}
which is exactly the classical Lorenz system, with the identifications:
\[
X \leftrightarrow E, \quad 
Y \leftrightarrow P, \quad 
Z \leftrightarrow N.
\]
For appropriate parameter ranges 
(e.g.\ $\sigma = 10$, $r = 28$, $b = 8/3$),
the system exhibits the well-known butterfly attractor.

Thus a laser with field--polarization coupling can display the full range of
Lorenz-type behavior: fixed points, limit cycles, period doubling,
intermittency, and fully developed chaos.

\subsection{Other Mechanisms for Chaos in Lasers}

Beyond the Lorenz structure, there are several other standard routes to chaos
in laser physics:

\paragraph{1. Delayed optical feedback (Ikeda-type laser).}
The output of the laser is partially reflected back after a time delay $\tau$.
This produces a delay-differential equation,
\[
\dot x(t) = F\big(x(t), x(t-\tau)\big),
\]
which is infinite-dimensional and easily becomes chaotic.  
This is a common model for chaotic diode lasers and mode-locked lasers.

\paragraph{2. External periodic modulation.}
Modulating pump strength or cavity loss,
\[
\dot x = x(y-1), \qquad \dot y = -\alpha y - \alpha x y + \alpha (\beta + A\cos\omega t),
\]
creates a periodically forced nonlinear oscillator.  
Poincaré sections reduce this to a 2D map, where period doubling cascades and
chaos readily occur.

\paragraph{3. Injection locking (two-laser coupling).}
With injected field $E_\text{inj}$, the laser evolution includes an additional
phase equation.  The resulting 3D system can undergo:
\begin{itemize}
\item quasiperiodicity,
\item torus breakdown,
\item Shilnikov chaos.
\end{itemize}

\paragraph{4. Multimode lasers.}
Allowing multiple longitudinal or transverse modes introduces additional field
amplitudes:
\[
E_1, E_2, \dots
\]
so even two-mode lasers lead to 4 or more state variables and chaotic dynamics.

\paragraph{5. Reduction to discrete maps.}
The laser system may be sampled stroboscopically (Poincaré map) or reduced to a
1D or 2D discrete map under suitable slow–fast assumptions.  
Typical examples include logistic-type maps,
\[
x_{n+1} = r x_n (1 - x_n),
\]
arising in periodically pumped lasers.

\subsection{Summary Chaotic Laser}

While the basic two-dimensional rate equations cannot display chaos, even
minimal physical extensions---such as including polarization, phase, time-delay,
periodic forcing, or multimode structure---produce systems that are
mathematically equivalent to classical chaotic models, most famously the
Lorenz system.


\end{document}
