\documentclass[11pt,a4paper]{article}
\usepackage[margin=2.5cm]{geometry}
\usepackage{amsmath,amssymb}
\usepackage{enumitem}
\usepackage{hyperref}
\usepackage{graphicx}
\usepackage{xcolor}
\usepackage{listings}

\title{Modeling Tutorial: From One-Dimensional Growth to Predator--Prey Systems}
\author{Prof.\ Robert J. Flassig}
\date{\today}

\lstset{
  basicstyle=\ttfamily\footnotesize,
  backgroundcolor=\color{gray!10},
  frame=single,
  breaklines=true,
  postbreak=\mbox{\textcolor{red}{$\hookrightarrow$}\space},
  keywordstyle=\color{blue},
  commentstyle=\color{gray!70}\itshape
}

% --- No paragraph indent, small skip between paragraphs ---
\setlength{\parindent}{0pt}
\setlength{\parskip}{0.8em}

\newcommand{\dd}{\mathrm{d}}

\begin{document}
\maketitle

\section{Learning Goals}
By the end of this tutorial you will be able to
\begin{itemize}[leftmargin=*,itemsep=0.3em]
  \item translate verbal assumptions into quantitative models,
  \item simulate and visualize the dynamics of nonlinear systems,
  \item interpret the meaning of model parameters in a physical or ecological context,
  \item explore how parameter changes affect qualitative system behavior,
  \item optionally: perform stability and phase-space analysis.
\end{itemize}

\section{The Logistic Growth Model (1D)}
We start with a single variable $x(t)$ representing, for example, a population size or concentration.

\subsection{Model construction}
\textbf{(Verbal) Assumptions:}
\begin{enumerate}[label=\textbf{A\arabic*},leftmargin=*]
  \item The population grows at a rate proportional to its current size.
  \item Growth slows down as the population approaches a maximum capacity $K$.
\end{enumerate}


\textbf{Modeling Verbal Assumptions:}
\[
\frac{\dd x}{\dd t} = r\,x\!\left(1-\frac{x}{K}\right),
\]
where
\begin{itemize}[leftmargin=2em,itemsep=0.2em]
  \item $r$: intrinsic growth rate [1/time],
  \item $K$: carrying capacity,
  \item $x$: population size.
\end{itemize}

\textbf{Interpretation:}  
At small $x$, the term $(1 - x/K)\approx1$, and we recover exponential growth.  
As $x\to K$, the growth term approaches zero—resources limit further increase.

\subsection{Numerical simulation in Python/Colab}
Below is an example of a minimal simulation using the \texttt{Euler} method:

\begin{lstlisting}[language=Python,caption={Simple Euler integration of the logistic model}]
import numpy as np
import matplotlib.pyplot as plt

# Parameters
r = 1.0     # growth rate
K = 10.0    # carrying capacity
x0 = 0.5    # initial population
dt = 0.01   # time step
T  = 10.0   # total time

# Time vector and storage
t = np.arange(0, T, dt)
x = np.zeros_like(t)
x[0] = x0

# Euler integration
for i in range(1, len(t)):
    dxdt = r * x[i-1] * (1 - x[i-1]/K)
    x[i] = x[i-1] + dt * dxdt

# Plot
plt.plot(t, x)
plt.xlabel("time t")
plt.ylabel("population x")
plt.title("Logistic growth")
plt.show()
\end{lstlisting}

\subsection{Tasks}
\begin{enumerate}[label=\textbf{T\arabic*},leftmargin=*]
  \item Interpret the parameters $r$ and $K$: what does changing each do biologically or physically?
  \item Re-run the simulation for several $r$ values. How does the system’s approach to equilibrium change?
  \item Add an initial condition $x_0 > K$ and interpret the dynamics.
  \item Create a small plot grid comparing different $r$ and $K$ combinations.
  \item \textbf{Optional:} Determine analytically the equilibrium points and their stability.
\end{enumerate}

\section{The Predator--Prey System (2D)}
Now consider a coupled system of two interacting populations:
\begin{align}
\frac{\dd x}{\dd t} &= r\,x\!\left(1-\frac{x}{K}\right) - \alpha x y,\\
\frac{\dd y}{\dd t} &= \beta x y - \delta y.
\end{align}

\subsection{Meaning of parameters}
\begin{itemize}[leftmargin=2em,itemsep=0.2em]
  \item $r$: prey reproduction rate,
  \item $K$: prey carrying capacity,
  \item $\alpha$: predation rate coefficient,
  \item $\beta$: efficiency with which prey consumption translates to predator growth,
  \item $\delta$: predator death rate.
\end{itemize}

\subsection{Initial code example}
The same Euler idea extends easily:

\begin{lstlisting}[language=Python,caption={Predator-prey simulation (Euler scheme)}]
# Parameters
r, K = 1.0, 5.0
alpha, beta, delta = 0.2, 0.1, 0.4
x0, y0 = 1.0, 0.5
dt, T = 0.01, 100.0

t = np.arange(0, T, dt)
x, y = np.zeros_like(t), np.zeros_like(t)
x[0], y[0] = x0, y0

for i in range(1, len(t)):
    dx = r*x[i-1]*(1 - x[i-1]/K) - alpha*x[i-1]*y[i-1]
    dy = beta*x[i-1]*y[i-1] - delta*y[i-1]
    x[i] = x[i-1] + dt*dx
    y[i] = y[i-1] + dt*dy

plt.figure(figsize=(10,4))
plt.subplot(1,2,1)
plt.plot(t, x, label="prey")
plt.plot(t, y, label="predator")
plt.xlabel("time"); plt.legend()

plt.subplot(1,2,2)
plt.plot(x, y)
plt.xlabel("prey x"); plt.ylabel("predator y")
plt.title("Phase space")
plt.tight_layout(); plt.show()
\end{lstlisting}

\subsection{Tasks}
\begin{enumerate}[label=\textbf{P\arabic*},leftmargin=*]
  \item Interpret each parameter in words. Which ones affect prey equilibrium? Which influence oscillation amplitude?
  \item Run the model for baseline parameters. Describe the qualitative time evolution.
  \item Produce a phase-space plot $(x,y)$. What kind of trajectory emerges?
  \item Vary one parameter systematically (e.g.\ $\alpha$ or $\delta$) and discuss changes in the qualitative behavior.
  \item \textbf{Optional:} Add nullcline plots and use arrows to indicate direction fields.
\end{enumerate}

\section{Mini Projects}
\subsection{Project 1: Logistic growth with harvesting}
Add a constant harvest term $h$:
\[
\frac{\dd x}{\dd t} = r\,x\!\left(1-\frac{x}{K}\right) - h.
\]
\textbf{Questions:}
\begin{enumerate}[label=\alph*.,leftmargin=*]
  \item For what $h$ does the population go extinct? 
  \item Simulate several $h$ values and visualize $x(t)$.
  \item Estimate the critical $h_c$ separating survival and extinction.
  \item Discuss management implications (sustainable yield).
\end{enumerate}

\subsection{Project 2: Predator--prey with saturation}
Use the Holling's type II response:
\[
\frac{\dd x}{\dd t} = r\,x\!\left(1-\frac{x}{K}\right) - \frac{\alpha x}{1 + h x}y, \quad
\frac{\dd y}{\dd t} = \beta x y - \delta y.
\]
\textbf{Tasks:}
\begin{enumerate}[label=\alph*.,leftmargin=*]
  \item Implement the model and vary $h$.
  \item Compare time series and phase-space trajectories.
  \item Interpret biologically how the handling time $h$ changes stability.
  \item \textbf{Optional:} derive the nullclines and discuss stability.
\end{enumerate}

\section{Supporting Hints for Simulation}
\begin{itemize}[leftmargin=*,itemsep=0.25em]
  \item Reuse and modify the \textbf{Google Colab templates} provided in class.  
        They contain ready-made plotting cells and Euler integration loops.
  \item When in doubt, test your code with small time steps (\texttt{dt = 0.001}) and compare to larger ones.
  \item Experiment with \texttt{scipy.integrate.solve\_ivp} for higher accuracy.
  \item Use modern AI assistants (\textbf{ChatGPT}, \textbf{Claude.ai}, etc.) to:
        \begin{itemize}
          \item debug code and syntax errors,
          \item generate plotting routines or parameter sweep loops,
          \item check your understanding of equations and their derivatives.
        \end{itemize}
  \item Always document AI-assisted contributions and verify numerical results independently.
  \item Comment each code cell clearly: what parameters did you change and why?
\end{itemize}

\section{Deliverables}
Your short report (2--4 pages) should include:
\begin{itemize}[leftmargin=*,itemsep=0.25em]
  \item Equations and parameter definitions.
  \item Plots of time series and phase-space trajectories.
  \item Interpretation of how parameters affect dynamics.
  \item \textbf{Optional:} analytical or stability results.
  \item Reflection on modeling choices and learning outcomes.
\end{itemize}

\section{General Hints}
\begin{itemize}[leftmargin=*,itemsep=0.25em]
  \item Use dimensionless time $\tau = r t$ to reduce parameter count.
  \item Normalize populations by $K$ to simplify visualization.
  \item If results diverge or oscillate unnaturally, reduce $\Delta t$.
  \item For visual clarity, annotate phase-space arrows with \texttt{quiver()} or \texttt{streamplot()}.
\end{itemize}

\end{document}
